\section{Proposed protections against FIAs}


%%%%%%%%%%%%%%%%%%%%%%%%%%%%%%%%%%%%%%%%%%%%%%%%%%%%%%%%%%%%%%%%%%%%%%%%%%%%%%%%%%%%%%%%%%%%%%%%%%%%%%%%%%%%
\begin{frame}{Introduction}
    \begin{block}{Protections}
        \begin{itemize}
            \item We propose 3 lightweight countermeasures using parity codes
            \begin{itemize}
                \item Simple parity
                \item Hamming Code
                \item Hamming Code with an additional bit (SECDED)
            \end{itemize}
        \end{itemize}
    \end{block}
\end{frame}
%%%%%%%%%%%%%%%%%%%%%%%%%%%%%%%%%%%%%%%%%%%%%%%%%%%%%%%%%%%%%%%%%%%%%%%%%%%%%%%%%%%%%%%%%%%%%%%%%%%%%%%%%%%%
\subsection{Simple Parity}
    \begin{frame}{Detection of single-bit errors - Simple Parity}
        \begin{block}{}
            \begin{itemize}
                \justifying
                \item Focusing into lightweight protections for small systems
                \item Often used for error detection.
                \item Add an extra bit for parity computation.
                \item Can only detect one error without correction.
            \end{itemize}
        \end{block}

        \vfill
        
        \begin{figure}
            \centering
            \includegraphics[width=.5\textwidth, page=1]{src/3_strategies/img/simple_parity.pdf}
            \caption{Simple parity codeword}
            \label{fig:simple_parity_codeword}
        \end{figure}
    \end{frame}

    \begin{frame}{Implementation - Simple Parity}
        \begin{figure}
            \centering
            \includegraphics[width=\textwidth, page=1]{src/3_strategies/img/archi_contremesures.pdf}
            \caption{Simple Parity implementation}
            \label{fig:simple_parity_implem}
        \end{figure}
    \end{frame}
%%%%%%%%%%%%%%%%%%%%%%%%%%%%%%%%%%%%%%%%%%%%%%%%%%%%%%%%%%%%%%%%%%%%%%%%%%%%%%%%%%%%%%%%%%%%%%%%%%%%%%%%%%%%
\subsection{Hamming Code}
    \begin{frame}{Detection and correction of single-bit errors - Hamming Code}
        \begin{block}{}
            \begin{itemize}
                \justifying
                \item Linear error-correcting codes invented by Richard W. Hamming~\cite{H-50-bstj}.
                \item Mostly used in digital communication and data storage systems.
                \item Detect and correct single-bit error.
                \item Redundancy bits are placed in power of 2 positions.
                \item The number of redundancy bits depends on the number of bits to be protected {\scriptsize ($ 2^r \ge m + r + 1 $)}
            \end{itemize}
        \end{block}

        \begin{minipage}[c]{0.4\linewidth}
            \begin{equation} \label{equat:hamming_encoder}
                \begin{split}
                    r_{0} &= d_{0} \oplus d_{1} \oplus d_{3} \oplus d_{4} \oplus d_{6} \\
                    r_{1} &= d_{0} \oplus d_{2} \oplus d_{3} \oplus d_{5} \oplus d_{6} \\
                    r_{2} &= d_{1} \oplus d_{2} \oplus d_{3} \\
                    r_{3} &= d_{4} \oplus d_{5} \oplus d_{6}
                \end{split}
            \end{equation}
        \end{minipage}\hfill%
        \begin{minipage}[c]{0.55\linewidth}
            \begin{figure}
                \centering
                \includegraphics[width=\textwidth, page=1]{src/3_strategies/img/hamming_bit.pdf}
                \caption{Hamming codeword}
                \label{fig:hamming_codeword}
            \end{figure}
        \end{minipage}
    \end{frame}
    
    \begin{frame}{Implementation - Hamming Code}
        \begin{figure}
            \centering
            \includegraphics[width=.9\textwidth, page=2]{src/3_strategies/img/archi_contremesures.pdf}
            \caption{Hamming Code implementation for independant registers}
            \label{fig:hamming_code_implem_independant_register}
        \end{figure}
    \end{frame}
    
    \begin{frame}{Implementation - Hamming Code}
        \begin{figure}
            \centering
            \includegraphics[width=.9\textwidth, page=3]{src/3_strategies/img/archi_contremesures.pdf}
            \caption{Hamming Code implementation for Register File}
            \label{fig:hamming_code_implem_rf}
        \end{figure}
    \end{frame}
%%%%%%%%%%%%%%%%%%%%%%%%%%%%%%%%%%%%%%%%%%%%%%%%%%%%%%%%%%%%%%%%%%%%%%%%%%%%%%%%%%%%%%%%%%%%%%%%%%%%%%%%%%%%
\subsection{Hamming Code - SECDED}
\begin{frame}{Detection of two-bit errors and correction of single-bit errors - SECDED}
    \begin{block}{}
        \begin{itemize}
            \justifying
            \item Based on Hamming Code.
            \item Detect two-bit error and correct single-bit error.
            \item The number of redundancy bits depends on the number of bits to be protected {\scriptsize ($ 2^r \ge m + r + 1 $)}.
            \item An additional bit is placed at index 0, it is called: general parity bit
        \end{itemize}
    \end{block}

    \begin{minipage}[c]{0.4\linewidth}
        \begin{equation} \label{equat:secded_encoder}
            \begin{split}
                r_{0}  &= d_{0} \oplus d_{1} \oplus d_{3} \oplus d_{4} \oplus d_{6} \\
                r_{1}  &= d_{0} \oplus d_{2} \oplus d_{3} \oplus d_{5} \oplus d_{6} \\
                r_{2}  &= d_{1} \oplus d_{2} \oplus d_{3} \\
                r_{3}  &= d_{4} \oplus d_{5} \oplus d_{6} \\
                gp_{0} &= \bigoplus_{i=0}^{6} d_{i} \oplus \bigoplus_{j=0}^{3} r_{j}
            \end{split}
        \end{equation}
    \end{minipage}\hfill%
    \begin{minipage}[c]{0.6\linewidth}
        \begin{figure}
            \centering
            \includegraphics[width=\textwidth, page=1]{src/3_strategies/img/secded.pdf}
            \caption{SECDED codeword}
            \label{fig:secded_codeword}
        \end{figure}
    \end{minipage}
\end{frame}
    
\begin{frame}{Implementation - SECDED}
    \begin{figure}
        \centering
        \includegraphics[width=.8\textwidth, page=4]{src/3_strategies/img/archi_contremesures.pdf}
        \caption{SECDED implementation for independant registers}
        \label{fig:secded_implem_independant_register}
    \end{figure}
\end{frame}

\begin{frame}{Implementation - SECDED}
    \begin{figure}
        \centering
        \includegraphics[width=.9\textwidth, page=5]{src/3_strategies/img/archi_contremesures.pdf}
        \caption{SECDED implementation for Register File}
        \label{fig:secded_implem_rf}
    \end{figure}
\end{frame}
%%%%%%%%%%%%%%%%%%%%%%%%%%%%%%%%%%%%%%%%%%%%%%%%%%%%%%%%%%%%%%%%%%%%%%%%%%%%%%%%%%%%%%%%%%%%%%%%%%%%%%%%%%%%
\begin{frame}{Threat model}
    \begin{block}{Threat model}
        \begin{itemize}
            \item DIFT-related registers + protection-related registers
            \item Single bit-flip in one register
            \item Single bit-flip in two registers at two distinct clock cycles
            \item Single bit-flip in two registers at a given clock cycle
            \item Multi-bit faults in one register at a given clock cycle
            \item Multi-bit faults in two registers at a given clock cycle
        \end{itemize}
    \end{block}
\end{frame}
%%%%%%%%%%%%%%%%%%%%%%%%%%%%%%%%%%%%%%%%%%%%%%%%%%%%%%%%%%%%%%%%%%%%%%%%%%%%%%%%%%%%%%%%%%%%%%%%%%%%%%%%%%%%
\begin{frame}{Implemented strategies - Group composition}
    \begin{table}[t]
        \centering
        \caption{Grouping composition of implemented strategies}
        \label{tab:strategies_group}
        \begin{tabular}{@{}cccc@{}}
            \toprule
                       & Grouping composition                                 & Number of groups & Number of registers \\ \midrule
            Baseline   & --                                                   & --               & 55                  \\
            Strategy 1 & Minimisation of groups                               & 5                & 65                  \\
            Strategy 2 & Protection per pipeline stage                        & 7                & 69                  \\
            Strategy 3 & Protection per register                              & 24               & 103                 \\
            Strategy 4 & Protection per register with slicing of CSR & 38               & 131                 \\
            Strategy 5 & Coupling sliced registers                            & 39               & 133                 \\
            \bottomrule
        \end{tabular}
    \end{table}
\end{frame}
%%%%%%%%%%%%%%%%%%%%%%%%%%%%%%%%%%%%%%%%%%%%%%%%%%%%%%%%%%%%%%%%%%%%%%%%%%%%%%%%%%%%%%%%%%%%%%%%%%%%%%%%%%%%
\begin{frame}{Implemented strategies - details}
    \begin{table}[t]
        \centering
        \caption{Summary of DIFT-related protected registers -- taking SECDED}
        \label{tab:detail_strategies_group}
        \begin{tabular}{@{}cccccc@{}}
            \toprule
                       & \begin{tabular}[c]{@{}c@{}}Number of\\ protected bits\end{tabular} & \begin{tabular}[c]{@{}c@{}}Number of\\ redundancy bits\end{tabular} & \begin{tabular}[c]{@{}c@{}}Number of\\ parity bits\end{tabular} & \begin{tabular}[c]{@{}c@{}}Number of\\ bits\end{tabular} \\ \midrule
            Strategy 1 & 107                                                                & 25                                                                  & 5                                                               & 157                                                      \\
            Strategy 2 & 107                                                                & 30                                                                  & 7                                                               & 164                                                      \\
            Strategy 3 & 107                                                                & 64                                                                  & 24                                                              & 215                                                      \\
            Strategy 4 & 103                                                                & 101                                                                 & 38                                                              & 266                                                      \\
            Strategy 5 & 102                                                                & 114                                                                 & 39                                                              & 280                                                      \\
            \bottomrule
        \end{tabular}
    \end{table}
\end{frame}
%%%%%%%%%%%%%%%%%%%%%%%%%%%%%%%%%%%%%%%%%%%%%%%%%%%%%%%%%%%%%%%%%%%%%%%%%%%%%%%%%%%%%%%%%%%%%%%%%%%%%%%%%%%%