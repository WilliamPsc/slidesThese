\section{Conclusion and Perspectives}


%%%%%%%%%%%%%%%%%%%%%%%%%%%%%%%%%%%%%%%%%%%%%%%%%%%%%%%%%%%%%%%%%%%%%%%%%%%%
\subsection{Conclusion}
\begin{frame}{Conclusion}
    \begin{exampleblock}{}
        \centering
        How can we maintain maximum protection against software attacks in the presence of physical attacks?
    \end{exampleblock}
    
    \begin{columns}
        \begin{column}{.7\linewidth}
            \begin{block}{Presented:}
                \begin{itemize}
                    \setbeamertemplate{itemize items}[triangle]
                    \item<1> Vulnerability assessment of a DIFT mechanism against FIA
                        \only<1>{
                            \begin{itemize}
                                \item We have shown that the DIFT mechanism is vulnerable
                                \item Presented different fault models adapted from the state-of-the-art to defeat the DIFT and its protections
                            \end{itemize}
                        }
                    \item<2> Open-Source tool to help find vulnerabilities during the conceptual phase. It enables the concept of \textit{Security by Design}
                    \item<3> Proposition of 3 lightweight countermeasures with a low overhead of area and performances
                        \only<3>{
                            \begin{itemize}
                                \item countermeasures based on parity codes
                                \item area overhead smaller than 8\%
                                \item no impact on performances
                            \end{itemize}
                        }
                \end{itemize}
            \end{block}
        \end{column}
        \begin{column}{.3\linewidth}
            \begin{figure}
                \centering
                \includegraphics[height=.25\textheight]{src/6_conclusion/img/conclusion.png}
            \end{figure}
        \end{column}
    \end{columns}
    
    \note{
        \textit{Vulnerability assessment:} taking into account different fault models up to very complex ones

        \textit{Countermeasures:} good efficiency, overhead is minimal compared to others countermeasures like BCH, spatial redundancy (duplication, triplication)
    }
\end{frame}

%%%%%%%%%%%%%%%%%%%%%%%%%%%%%%%%%%%%%%%%%%%%%%%%%%%%%%%%%%%%%%%%%%%%%%%%%%%%
\subsection{Perspectives}
\begin{frame}{Perspectives}
    \begin{block}{Short terms}
        \begin{itemize}
            \setbeamertemplate{itemize items}[triangle]
            \item<1> Propose more robust countermeasures to correct multiple faults
                \begin{itemize}
                    \item Evaluating of Reed-Solomon, BCH codes, or triplication
                    \item Evaluation of these countermeasures in terms of area and performances overhead compared to our actual proposed solutions
                \end{itemize}
            \item<2> Further development of FISSA
                \begin{itemize}
                    \item Better integration in the design workflow
                    \item More fault models
                    \item More configurability, for example, automatisation for finding targets
                    \item Adding a graphical user interface to provide better experience
                \end{itemize}
        \end{itemize}
    \end{block}
    
    \begin{figure}
        \centering
        \includegraphics[height=.25\textheight]{src/6_conclusion/img/perspectives.jpg}
    \end{figure}
\end{frame}
%%%%%%%%%%%
\begin{frame}{Perspectives}
    \begin{block}{Long terms}
        \begin{itemize}
            \setbeamertemplate{itemize items}[triangle]
            \item Conduct real-world FIA
                \begin{itemize}
                    \item Evaluation against clock glitches (Chip Whisperer~\cite{chipwhisperer}) or EMFI (Chip Shouter~\cite{chipshouter}) for examples.
                \end{itemize}
            \item Extend the assessment of more complex DIFT
                \begin{itemize}
                    \item Evaluation of DIFT with more bits in the tag (e.g: Raksha~\cite{DKK-07-sigarch} : 4-bit tags)
                    \item Evaluation of our proposed protections for these DIFT and comparison with other protections
                \end{itemize}
        \end{itemize}
    \end{block}

    \begin{figure}
        \centering
        \includegraphics[height=.25\textheight]{src/6_conclusion/img/perspectives.jpg}
    \end{figure}
\end{frame}
%%%%%%%%%%%%%%%%%%%%%%%%%%%%%%%%%%%%%%%%%%%%%%%%%%%%%%%%%%%%%%%%%%%%%%%%%%%%