\section{D-RI5CY -- Vulnerability Assessment}

%%%%%%%%%%%%%%%%%%%%%%%%%%%%%%%%%%%%%%%%%%%%%%%%%%%%%%%%%%%%%%%%%%%%%%%%%%%%
\subsection{D-RI5CY - origins and architecture}
\begin{frame}{D-RI5CY - origins}
    \begin{itemize}
        \item Design~\cite{PDGLC-18-hpec} made by researchers at Columbia University (USA) with Politecnico di Torino (Italy)
        \item Based on the 32-bit RISC-V processor: RI5CY (Pulp Platform)
        \item Open source\footnote{\url{https://github.com/sld-columbia/riscv-dift}}
        \item 1-bit tag datapath
        \item Flexible security policy that can be modified at runtime
    \end{itemize}

    \centering
    \vfill
    \includegraphics[height=1cm]{img/logo/riscv.png}
    \hspace{1cm}
    \includegraphics[height=1cm]{img/logo/pulp_logo.pdf}
    \vfill
\end{frame}
%%%%%%%%%%%%%%%%%%%%%%%%%%%%%%%%%%%%%%%%%%%%%%%%%%%%%%%%%%%%%%%%%%%%%%%%%%%%
\begin{frame}{D-RI5CY - architecture}
    \begin{figure}
        \centering
        \includegraphics[width=.9\textwidth]{src/2_vuln_assessment/img/RI5CY.pdf}
        \caption{Architecture of the D-RI5CY.}
        \label{fig:riscy}
    \end{figure}
\end{frame}
%%%%%%%%%%%%%%%%%%%%%%%%%%%%%%%%%%%%%%%%%%%%%%%%%%%%%%%%%%%%%%%%%%%%%%%%%%%%
\subsection{Vulnerability assessment}
\begin{frame}{Vulnerability Assessment}
    \begin{block}{Threat model}
        We consider an attacker able to:
        \begin{itemize}
            \item perform a physical attack to defeat the DIFT mechanism and realise a software attack,
            \item inject faults in DIFT-related registers:
                  \begin{itemize}
                      \item bit set,
                      \item bit reset,
                      \item bit-flip.
                  \end{itemize}
        \end{itemize}
    \end{block}

    % \begin{block}{Methodology}
    %     \begin{itemize}
    %         \item Analysis of 2 attacks use cases: buffer overflow attack, format string attack
    %         \item Analysis of 1 use case to complete the DIFT surface: compare/compute
    %     \end{itemize}
    % \end{block}
\end{frame}
%%%%%%%%%%%%%%%%%%%%%%%%%%%%%%%%%%%%%%%%%%%%%%%%%%%%%%%%%%%%%%%%%%%%%%%%%%%%
\subsection{Use case : presentation}
\begin{frame}{Case 1: Buffer overflow}
    \begin{itemize}
        \item The attacker exploits a buffer overflow to access the return address register ($RA$).
    \end{itemize}

    \begin{figure}
        \centering
        \begin{subfigure}[l]{.45\textwidth}
            \centering
            \includegraphics[width=.9\textwidth, page=1]{src/2_vuln_assessment/img/buffer_overflow/schemaPedagogique.pdf}
            \caption{Malicious buffer and $RA$ trusted}
            \label{fig:bo_1st_step}
        \end{subfigure}
        \begin{subfigure}[r]{.45\textwidth}
            \centering
            \includegraphics[width=.9\textwidth, page=5]{src/2_vuln_assessment/img/buffer_overflow/schemaPedagogique.pdf}
            \caption{Overflow and overwriting of $RA$ and its tag}
            \label{fig:bo_3rd_step}
        \end{subfigure}
    \end{figure}

    \begin{itemize}
        \item As the data in the source buffer is manipulated by the user, it is marked as \textcolor{red}{\textit{untrusted}}.
        \item Thanks to DIFT, the tags associated with the source buffer data overwrite the $RA$ register tag.
        \item When the function returns, the corrupted register $RA$ is loaded into $PC$ using a jalr instruction.
    \end{itemize}
\end{frame}

\begin{frame}
    \begin{figure}
        \centering
        \includegraphics[width=.75\textwidth]{src/2_vuln_assessment/img/buffer_overflow/bufferOverflowAttack_short.pdf}
        \caption{Temporal analysis of the tags propagation in \textit{Buffer Overflow} attack}
        \label{fig:analyseTempoBufferOverflow}
    \end{figure}
\end{frame}

\begin{frame}
    \begin{figure}
        \centering
        \includegraphics[height=.85\textheight]{src/2_vuln_assessment/img/buffer_overflow/arborescence_bufferOverflow.pdf}
        \caption{Logical analysis of the tags propagation in a \textit{Buffer Overflow} attack}
        \label{fig:analyseLogiqueBufferOverflow}
    \end{figure}
\end{frame}
%%%%%%%%%%%%%%%%%%%%%%%%%%%%%%%%%%%%%%%%%%%%%%%%%%%%%%%%%%%%%%%%%%%%%%%%%%%%%%%%%%%%%%%%%%%%%%%%%%%%%%%%%%%%
\begin{frame}[fragile]{Case 2: WU-FTPd}
    \begin{itemize}
        \justifying
        \item The vulnerability is the use of an unchecked user input as the format string parameter in functions that perform formatting, e.g. printf()
        \item An attacker can use the format tokens, to write into arbitrary locations of memory, e.g. the return address of the function.
    \end{itemize}

    % \vspace{1cm}
    % \hspace{2cm}
    \centering
    \begin{minipage}[c]{\textwidth}
        \begin{lstlisting}[language=C,label=code:printfNFormat]
void echo(){
    int a;
    register int i asm("x8");
    a = i;
    printf("%224u%n%35u%n%253u%n%n", 1, (int*) (a-4), 1, (int*) (a-3), 1, (int*) (a-2), (int*) (a-1));
}\end{lstlisting}
    \end{minipage}
\end{frame}

\begin{frame}
    \begin{figure}
        \centering
        \includegraphics[height=.85\textheight]{src/2_vuln_assessment/img/wuftpd/ftpd_short.pdf}
        \caption{Temporal analysis of the tags propagation in a \textit{format string} attack}
        \label{fig:analyseTempoFormatString}
    \end{figure}
\end{frame}

\begin{frame}
    \begin{figure}
        \centering
        \includegraphics[height=.85\textheight]{src/2_vuln_assessment/img/wuftpd/arborescence_wuftpd.pdf}
        \caption{Logical analysis of the tags propagation in a \textit{format string} attack}
        \label{fig:analyseLogiqueFormatString}
    \end{figure}
\end{frame}
%%%%%%%%%%%%%%%%%%%%%%%%%%%%%%%%%%%%%%%%%%%%%%%%%%%%%%%%%%%%%%%%%%%%%%%%%%%%%%%%%%%%%%%%%%%%%%%%%%%%%%%%%%%%
\subsection{Experimental Setup}
%%%%%%%%%%%%%%%%%%%%%%%%%%%%%%%%%%%%%%%%%%%%%%%%%%%%%%%%%%%%%%%%%%%%%%%%%%%%%%%%%%%%%%%%%%%%%%%%%%%%%%%%%%%%
\begin{frame}{Experimental Setup - Simulation fault injections campaign}
    \begin{itemize}
        \justifying
        \item Logical fault injection simulation is used for preliminary evaluations
              \begin{itemize}
                  \justifying
                  \item faults are injected in the HDL code at cycle accurate and bit accurate level
                  \item a set of 55 DIFT-related registers are targeted
                  \item a reference simulation is done without fault
                  \item results are classed in four groups
                        \begin{itemize}
                            \justifying
                            \item crash: reference cycle count exceeded,
                            \item silent: current faulted simulation is the same as the reference simulation
                            \item delay: illegal instruction is delayed
                            \item success: DIFT has been bypassed
                        \end{itemize}
              \end{itemize}
        \item Simulations with QuestaSim 10.6e.
    \end{itemize}
\end{frame}
%%%%%%%%%%%%%%%%%%%%%%%%%%%%%%%%%%%%%%%%%%%%%%%%%%%%%%%%%%%%%%%%%%%%%%%%%%%%%%%%%%%%%%%%%%%%%%%%%%%%%%%%%%%%
\begin{frame}{Main results : 3 cases}
    \begin{table}[H]
        \centering
        \caption{End of simulation status}
        \label{table:end_sim_by_status}
        \begin{tabular}{lrrrlr}
            \toprule
                            & Crash & NSTR & Delay & Success     & Total \\
            \midrule
            Buffer overflow & 0     & 1380 & 20    & 22 (1.55\%) & 1422  \\
            WU-FTPd         & 0     & 1767 & 77    & 52 (2.74\%) & 1896  \\
            Compare/Compute & 0     & 917  & 12    & 19 (2.00\%) & 948   \\
            \bottomrule
        \end{tabular}
    \end{table}
\end{frame}

\begin{frame}{Buffer overflow}
    \begin{table}[]
        \centering
        \caption{Buffer overflow : Register sensitivity as determined by fault model and simulation time}
        \label{table:end_sim_from_time_fault_register_buffer_overflow}
        \resizebox{\textwidth}{!}{%
            \begin{tabular}{llllllllllllllll}
                \toprule
                                                & \multicolumn{3}{c}{Cycle 3428} & \multicolumn{3}{c}{Cycle 3429} & \multicolumn{3}{c}{Cycle 3430} & \multicolumn{3}{c}{Cycle 3431} & \multicolumn{3}{c}{Cycle 3432}                                                                                                                 \\\cmidrule(lr){2-4}\cmidrule(lr){5-7}\cmidrule(lr){8-10}\cmidrule(lr){11-13}\cmidrule(lr){14-16}
                                                & set0                           & set1                           & bitflip                        & set0                           & set1                           & bitflip    & set0       & set1 & bitflip    & set0       & set1 & bitflip    & set0       & set1 & bitflip    \\
                \midrule
                pc\_if\_o\_tag                  &                                &                                &                                &                                &                                &            &            &      &            & \checkmark &      & \checkmark &            &      &            \\
                rf\_reg[1]                      &                                &                                &                                &                                &                                &            & \checkmark &      & \checkmark &            &      &            &            &      &            \\
                tcr\_q                          & \checkmark                     &                                &                                & \checkmark                     &                                &            & \checkmark &      &            & \checkmark &      &            & \checkmark &      &            \\
                \rowcolor{LightGray} tcr\_q[21] &                                &                                & \checkmark                     &                                &                                & \checkmark &            &      & \checkmark &            &      & \checkmark &            &      & \checkmark \\
                tpr\_q                          & \checkmark                     & \checkmark                     &                                & \checkmark                     & \checkmark                     &            &            &      &            &            &      &            &            &      &            \\
                \rowcolor{LightGray} tpr\_q[12] &                                &                                & \checkmark                     &                                &                                & \checkmark &            &      &            &            &      &            &            &      &            \\
                \rowcolor{LightGray} tpr\_q[15] &                                &                                & \checkmark                     &                                &                                & \checkmark &            &      &            &            &      &            &            &      &            \\
                \bottomrule
            \end{tabular}
        }
    \end{table}
\end{frame}

\begin{frame}{Discussion}
    \begin{itemize}
        \setbeamertemplate{itemize items}[triangle]
        \item 4212 simulations have been performed,
        \item 93 successes (2.21\%),
        \item We have shown that the D-RI5CY DIFT is vulnerable to FIA
        \item Propagation of faults is facilitated by paths fully made of \textit{AND} gates
    \end{itemize}
\end{frame}
%%%%%%%%%%%%%%%%%%%%%%%%%%%%%%%%%%%%%%%%%%%%%%%%%%%%%%%%%%%%%%%%%%%%%%%%%%%%%%%%%%%%%%%%%%%%%%%%%%%%%%%%%%%%