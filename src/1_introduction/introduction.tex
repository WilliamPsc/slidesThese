\section*{Introduction}

%%%%%%%%%%%%%%%%%%%%%%%%%%%%%%%%%%%%%%%%%%%%%%%%%%%%%%%%%%%%%%%%%%%%%%%%%%%%
\subsection{Context}
\begin{frame}{Context: Internet of Things}
    \begin{minipage}[c]{0.5\textwidth}
        \begin{block}{Internet of Things (IoT)}
            \begin{itemize}
                \setbeamertemplate{itemize items}[square]
                \justifying
                \item Wide range of application
                \item Fast growing market with exponential usage (on average 21 devices per household)
                \item Rely on sensors, depending on their use
                \item Collect and share data
                \item Manipulation of sensitive data
                \item Increasingly vulnerable to multiple threats
            \end{itemize}
        \end{block}
    \end{minipage}\hfill%
    \begin{minipage}[c]{0.5\textwidth}
        \begin{figure}
            \centering
            \includegraphics[width=.78\textwidth, trim={4.5cm 2.25cm 5.75cm 3.25cm}, clip]{src/1_introduction/img/iotapplications.pdf}
            % \caption{Example of IoT applications}
            \label{fig:iot_application}
        \end{figure}
        \vspace{-15pt}
        \begin{figure}
            \centering
            \includegraphics[width=.825\textwidth]{src/1_introduction/img/iot_forecasts.pdf}
            \vspace{-5pt}
            \caption{Number of IoT devices worldwide from 2022 to 2033 (from~\cite{statista_iot})}
            \label{fig:nbr_iot}
        \end{figure}
    \end{minipage}
\end{frame}

%%%%%%%%%%%%%%%%%%%%%%%%%%%%%%%%%%%%%%%%%%%%%%%%%%%%%%%%%%%%%%%%%%%%%%%%%%%%
% \begin{frame}{Context: IoT Under Threats}
%     \begin{figure}
%         \centering
%         \includegraphics[height=.85\textheight, trim={10.85cm 0cm 0cm 0cm}, clip]{src/1_introduction/img/threats_iot.pdf}
%         \caption{Graph from BitDefender~\cite{bitdefender_netgear}}
%     \end{figure}
% \end{frame}

\begin{frame}{Context: IoT Under Threats}
    \begin{block}{Threats}
        \begin{itemize}
            \setbeamertemplate{itemize items}[square]
            \justifying
            \item Software threats: malwares~\cite{FIMI-23-access}, memory overflow attacks~\cite{CWCBW-00-discex}, SQL injection, etc
            \item Network threats: Man-In-The-Middle~\cite{CDL-16-commsurtuto}, jamming~\cite{PZ-22-commsurtuto}, DDoS, etc
            \item Hardware threats: Reverse Engineering, Side-Channel Attacks (SCA)~\cite{DM-21-appiot}, Fault Injection Attacks (FIA)~\cite{BCNTW-06-procieee}
        \end{itemize}
    \end{block}

    \begin{center}
        \begin{minipage}[r]{.4\textwidth}
            \begin{figure}
                \centering
                \includegraphics[height=.62\textheight, trim={10.85cm 0cm 0cm 0cm}, clip]{src/1_introduction/img/threats_iot.pdf}
            \end{figure}
        \end{minipage}\hspace{.5cm}%
        \begin{minipage}[c]{0.3\textwidth}
            \captionof{figure}{Graph from BitDefender~\cite{bitdefender_netgear}}
        \end{minipage}
    \end{center}
\end{frame}

\begin{frame}{Context: IoT Under Threats}
    \begin{minipage}[c]{0.7\textwidth}
        \begin{block}{Threats}
            \begin{itemize}
                \setbeamertemplate{itemize items}[square]
                \justifying
                \item \textbf{\textcolor{red}{Software threats}}: malwares~\cite{FIMI-23-access}, memory overflow attacks~\cite{CWCBW-00-discex}, SQL injection, etc
                \item Network threats: Man-In-The-Middle~\cite{CDL-16-commsurtuto}, jamming~\cite{PZ-22-commsurtuto}, DDoS, etc
                \item \textbf{\textcolor{red}{Hardware threats}}: physical attacks such as reverse engineering, Side-Channel Attacks (SCA)~\cite{DM-21-appiot}, \underline{Fault Injection Attacks (FIA)}~\cite{BCNTW-06-procieee}
            \end{itemize}
        \end{block}
    \end{minipage}\hfill%
    \begin{minipage}[c]{0.3\textwidth}
        \begin{figure}
            \centering
            \includegraphics[width=.5\textwidth]{src/1_introduction/img/threats.png}
        \end{figure}
    \end{minipage}
\end{frame}
%%%%%%%%%%%%%%%%%%%%%%%%%%%%%%%%%%%%%%%%%%%%%%%%%%%%%%%%%%%%%%%%%%%%%%%%%%%%
\subsection{Software threats: Information Flow Tracking}
\begin{frame}{Software threats: Information Flow Tracking}
    \begin{minipage}[c]{0.55\textwidth}
        \begin{block}{}
            \begin{itemize}
                \setbeamertemplate{itemize items}[square]
                \justifying
                \item Security mechanism
                \item Protection against software attacks~\cite{HAK-21-acmcsur}% (e.g.: \textit{buffer overflow}, \textit{format string}, \textit{SQL injections}, \ldots)
                \item Static or \underline{\textbf{Dynamic}}
                \item Software, \underline{\textbf{Hardware}} or Hybrid
                      \only<1>{\item \textcolor{Gainsboro}{Hardware DIFT: off-core, off-loading core, in-core}}
                      \only<2->{\item Hardware DIFT:} \only<2>{\textbf{off-core} \cite{KDK-09-dsn}, \textcolor{Gainsboro}{off-loading core, in-core}}\only<3>{off-core, \textbf{off-loading core} \cite{CKSFGMRRRV-08-sigarch}, \textcolor{Gainsboro}{in-core}}\only<4>{off-core, off-loading core, \underline{\textbf{in-core}} \cite{DKK-07-sigarch}}
            \end{itemize}
        \end{block}
        \only<2>{
            \begin{exampleblock}{}
                \begin{itemize}
                    \setbeamertemplate{itemize items}[square]
                    \justifying
                    \item Advantage: no internal hardware modification to the main core.
                \end{itemize}
            \end{exampleblock}
            
            \begin{alertblock}{}
                \begin{itemize}
                    \setbeamertemplate{itemize items}[square]
                    \justifying
                    \item Disadvantage: needs support from the OS for the synchronization between data and tags.
                \end{itemize}
            \end{alertblock}
        }
        \only<3>{
            \begin{exampleblock}{}
                \begin{itemize}
                    \setbeamertemplate{itemize items}[square]
                    \justifying
                    \item Advantage: hardware does not need to know DIFT tags and policies, and no synchronization is needed.
                \end{itemize}
            \end{exampleblock}
            
            \begin{alertblock}{}
                \begin{itemize}
                    \setbeamertemplate{itemize items}[square]
                    \justifying
                    \item Disadvantage: requires a multicore CPU, reducing the number of cores available and increase the power consumption.
                \end{itemize}
            \end{alertblock}
        }
        \only<4>{
            \begin{exampleblock}{}
                \begin{itemize}
                    \setbeamertemplate{itemize items}[square]
                    \justifying
                    \item Advantage: no multicore CPU and no synchronization are needed. Very low performances overhead.
                \end{itemize}
            \end{exampleblock}
            
            \begin{alertblock}{}
                \begin{itemize}
                    \setbeamertemplate{itemize items}[square]
                    \justifying
                    \item Disadvantage: highly invasive modifications of internal hardware for tags computations and storing.
                \end{itemize}
            \end{alertblock}
        }
    \end{minipage}\hfill%
    \begin{minipage}[c]{0.4\textwidth}
        \only<1>{
            \begin{figure}
                \centering
                \includegraphics[width=\textwidth]{src/1_introduction/img/arborescence_ift.pdf}
                \caption{Taxonomy of IFTs}
                \label{fig:taxoDIFT}
            \end{figure}
        }
        \only<2>{
            \begin{figure}
                \centering
                \includegraphics[height=.5\textheight]{src/1_introduction/img/offcore.pdf}
                \caption{Representation of a Hardware Off-Core DIFT (inspired by~\cite{KDK-09-dsn})}
                \label{fig:offcore}
            \end{figure}
        }
        \only<3>{
            \begin{figure}
                \centering
                \includegraphics[height=.5\textheight]{src/1_introduction/img/offloading.pdf}
                \caption{Representation of a Hardware Off-Loading DIFT (inspired by~\cite{KDK-09-dsn})}
                \label{fig:offloading}
            \end{figure}
        }
        \only<4>{
            \begin{figure}
                \centering
                \includegraphics[width=\textwidth]{src/1_introduction/img/incore.pdf}
                \caption{Representation of a Hardware In-Core DIFT (inspired by~\cite{KDK-09-dsn})}
                \label{fig:incore}
            \end{figure}
        }
    \end{minipage}
\end{frame}

\begin{frame}{Dynamic Information Flow Tracking}
    \begin{minipage}[c]{0.35\linewidth}
        \begingroup
        \begin{block}{\textbf{Three} steps}
            \begin{itemize}
                \item<1-> Tag initialisation
                \item<2-> Tag propagation
                \item<3-> Tag check
            \end{itemize}
        \end{block}
        \endgroup
    \end{minipage}\hfill%
    \begin{minipage}[c]{0.6\linewidth}
        \begin{figure}
            \centering
            \only<1>{\includegraphics<1>[width=\textwidth, page=1]{src/1_introduction/img/schemaDIFT.pdf}}
            \only<2>{\includegraphics<2>[width=\textwidth, page=2]{src/1_introduction/img/schemaDIFT.pdf}}
            \only<3->{\includegraphics<3->[width=\textwidth, page=3]{src/1_introduction/img/schemaDIFT.pdf}}
            \label{fig:schemaDIFT}
        \end{figure}
    \end{minipage}
\end{frame}
%%%%%%%%%%%%%%%%%%%%%%%%%%%%%%%%%%%%%%%%%%%%%%%%%%%%%%%%%%%%%%%%%%%%%%%%%%%%
\subsection{Hardware threats: Physical Attacks}
\begin{frame}{Hardware threats: Physical Attacks}
    \begin{block}{}
        \begin{itemize}
            \setbeamertemplate{itemize items}[square]
            \justifying
            \item<1> Reverse Engineering: process of information retrieval from a product by analysing and understanding the design, functionality, and operation of existing hardware
            \item<2> Side-Channel Attacks: exploit information leakages on the circuit behaviour
            \item<3> \underline{Fault Injection Attacks}: involve deliberately introducing one or more fault(s) into the system to observe its behaviour and identify potential vulnerabilities.
        \end{itemize}
    \end{block}

    \only<1>{
        \begin{figure}
            \centering
            \includegraphics[height=.5\textheight, page=1]{src/1_introduction/img/rev_eng.png}
            \label{fig:revEng}
        \end{figure}
    }

    \only<2>{
        \begin{figure}
            \centering
            \includegraphics[height=.5\textheight, page=3]{src/1_introduction/img/physicalAttacks.pdf}
            \label{fig:sca}
        \end{figure}
    }

    \only<3>{
        \begin{figure}
            \centering
            \includegraphics[height=.5\textheight, page=2]{src/1_introduction/img/physicalAttacks.pdf}
            \label{fig:fia}
        \end{figure}
    }
\end{frame}
%%%%%%%%%%%%%%%%%%%%%%%%%%%%%%%%%%%%%%%%%%%%%%%%%%%%%%%%%%%%%%%%%%%%%%%%%%%%
\subsection{Motivations}
\begin{frame}{Motivations}
    \begin{block}{}
        \begin{itemize}
            % \item Possible to recover computed secret data using FIA on the RISC-V Rocket processor\footnote[frame]{\tiny\fullcite{LBDP-19-date}}
            \item Power supply : manipulations to control the program counter\footnote[frame]{\tiny\fullcite{TSW-16-fdtc}},
            \item EM Fault Injection (EMFI) : to recover an AES key by targeting the cache hierarchy and the MMU\footnote[frame]{\tiny\fullcite{TBELB-21-jce}},
            \item Laser Fault Injection (LFI) : allow the replay of instructions on a 32-bit microcontroller\footnote[frame]{\tiny\fullcite{KDD-21-dsd}}.
        \end{itemize}
    \end{block}
\end{frame}
%%%%%%%%%%%%%%%%%%%%%%%%%%%%%%%%%%%%%%%%%%%%%%%%%%%%%%%%%%%%%%%%%%%%%%%%%%%%
\subsection{Research challenge}
\begin{frame}{Research challenge}
    \begin{exampleblock}{}
        How can we maintain maximum protection against software attacks in the presence of physical attacks?
    \end{exampleblock}
\end{frame}
%%%%%%%%%%%%%%%%%%%%%%%%%%%%%%%%%%%%%%%%%%%%%%%%%%%%%%%%%%%%%%%%%%%%%%%%%%%%
\subsection{Objectives}
\begin{frame}{Objectives of this PhD Thesis}
    \begin{block}{}
        \begin{itemize}
            \setbeamertemplate{itemize items}[triangle]
            \justifying
            \item Provide a robust security mechanism against software and hardware threats,
            \item Propose lightweight countermeasures against FIA,
            \item Take into account constraints, such as area and performance overhead.
        \end{itemize}
    \end{block}
\end{frame}
%%%%%%%%%%%%%%%%%%%%%%%%%%%%%%%%%%%%%%%%%%%%%%%%%%%%%%%%%%%%%%%%%%%%%%%%%%%%